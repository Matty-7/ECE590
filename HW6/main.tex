\documentclass[12 pt]{article}         
\usepackage{amsfonts, amssymb}
\usepackage{fancyhdr}
\usepackage{amsmath}
\usepackage{fancyhdr}
\usepackage{verbatimbox}
\usepackage{tikz}
\usetikzlibrary{automata, positioning}

\oddsidemargin=-0.5cm                  
\setlength{\textwidth}{6.5in}          
\addtolength{\voffset}{-20pt}        		
\addtolength{\headsep}{25pt}

\setlength{\headheight}{27.2pt}
\addtolength{\topmargin}{-12.7pt}

\pagestyle{fancy}
\fancyhf{}
\fancyhead[L]{Theory and Practice of Algorithms \\ Homework 6}
\fancyhead[R]{Jingheng Huan \\ \today}
\fancyfoot[C]{\thepage}

\begin{document}

\section*{Problem 1}

(a) $2n^3 - 8n^2 + 32n + 9 \in O(n^3)$ is True. The dominant term in the polynomial is $2n^3$. As n approaches infinity, the lower-degree terms $(-8n^2 + 32n + 9)$ become negligible in comparison to $2n^3$. Also, we have: $2n^3 - 8n^2 + 32n + 9 \leq C \cdot n^3 \quad$ \text{for some constant } C \text{ and sufficiently large } n.


(b) $n^p \in O(e^n)$ , where  $p \in \mathbb{R}$  and  $p \geq 0$ is True. For any real  $p \geq 0$ ,  $n^p$  is a polynomial function, and  $e^n$  is an exponential function. Exponential functions grow faster than any polynomial function. When $n \to \infty$, we have $\lim_{n \to \infty} \frac{n^p}{e^n} = 0$. There exists a constant  $C > 0$  and  $n_0 > 0$  such that for all  $n > n_0$ , we have  $n^p \leq C \cdot e^n$. This implies that  $n^p$  grows slower than  $e^n$ . Therefore,  $n^p \in O(e^n)$ .







\vspace{1cm}

\section*{Problem 2}

(a) 
\textbf{Scan for ‘\#’}: Move right until the ‘\#’ symbol is found. If ‘\#’ is not found, reject.


\textbf{Compare symbols symmetrically}: Repeat until all symbols before '\#' are processed:
    \begin{enumerate}
        \item Move left from '\#' to find the first '0' or '1'), replace with 'X', and remember its value.
        \item Move right to '\#', then move right to find the first '0' or '1' after '\#'.
        \item Compare this symbol with the remembered value:
        \begin{itemize}
            \item If they match, mark it.
            \item If they do not match, then reject.
        \end{itemize}
    \end{enumerate}

\textbf{Check for extra symbols}: After all symbols before '\#' are marked, move right from '\#' to check for any unmarked symbols. If unmarked symbols are found, reject; else, accept.

\vspace{1cm}

(b)
\noindent The Turing machine \( M \) is defined as follows:


\( Q = \{ q_0, q_1, q_2, q_3, q_{\text{accept}}, q_{\text{reject}} \} \) is the set of states.


\( \Sigma = \{ 0, 1, \# \} \) is the input alphabet.


\( \Gamma = \{ 0, 1, \#, \_ \} \) is the tape alphabet, where \( \_ \) represents the blank symbol.


\( q_0 \) is the initial state.


\( q_{\text{accept}} \) is the accept state.


\( q_{\text{reject}} \) is the reject state.


\( \delta \) is the transition function, defined as follows:


\begin{center}
\begin{tabular}{|c|c|c|c|c|}
\hline
Current State & Current Symbol & New Symbol & Direction & New State \\
\hline
$q0$ & $0$   & $X$   & $R$ & $q1$ \\
$q0$ & $1$   & $X$   & $R$ & $q2$ \\
$q0$ & $\#$  & $\#$  & $R$ & $q6$ \\
\hline
$q1$ & $0$   & $0$   & $R$ & $q1$ \\
$q1$ & $1$   & $1$   & $R$ & $q1$ \\
$q1$ & $X$   & $X$   & $R$ & $q1$ \\
$q1$ & $\#$  & $\#$  & $R$ & $q3$ \\
\hline
$q2$ & $0$   & $0$   & $R$ & $q2$ \\
$q2$ & $1$   & $1$   & $R$ & $q2$ \\
$q2$ & $X$   & $X$   & $R$ & $q2$ \\
$q2$ & $\#$  & $\#$  & $R$ & $q5$ \\
\hline
$q3$ & $X$   & $X$   & $R$ & $q3$ \\
$q3$ & $0$   & $X$   & $L$ & $q4$ \\
\hline
$q4$ & $0$   & $0$   & $L$ & $q4$ \\
$q4$ & $1$   & $1$   & $L$ & $q4$ \\
$q4$ & $X$   & $X$   & $L$ & $q4$ \\
$q4$ & $\#$  & $\#$  & $L$ & $q4$ \\
$q4$ & $\_$  & $\_$  & $R$ & $q0$ \\
\hline
$q5$ & $X$   & $X$   & $R$ & $q5$ \\
$q5$ & $1$   & $X$   & $L$ & $q4$ \\
\hline
$q6$ & $X$   & $X$   & $R$ & $q6$ \\
$q6$ & $\_$  & $\_$  & $S$ & $q\_accept$ \\
\hline
\end{tabular}
\end{center}


\noindent\textbf{Collaborators: None}

\end{document}