\documentclass[12 pt]{article}         
\usepackage{amsfonts, amssymb}
\usepackage{fancyhdr}
\usepackage{amsmath}
\usepackage{tikz}
\usetikzlibrary{automata, positioning}

\oddsidemargin=-0.5cm                  
\setlength{\textwidth}{6.5in}          
\addtolength{\voffset}{-20pt}        		
\addtolength{\headsep}{25pt}

\setlength{\headheight}{27.2pt}
\addtolength{\topmargin}{-12.7pt}

\pagestyle{fancy}
\fancyhf{}
\fancyhead[L]{Theory and Practice of Algorithms \\ Homework 4}
\fancyhead[R]{Jingheng Huan \\ \today}
\fancyfoot[C]{\thepage}

\begin{document}

\section*{Problem 1}

(a) We proceed by defining separate CFGs for the two parts of the language and then combining them.


Let the CFG \( G = (V, \Sigma, R, S) \) be defined as follows:

Variables: \( V = \{ S, S_1, S_2, L_3, L_4, C, D \} \)


Terminals: \( \Sigma = \{ 0, 1, \# \} \)


Start Variable: \( S \)


Production Rules \( R \):

Start Rule:
   \[
   S \rightarrow S_1 \mid S_2
   \]

Productions for \( S_1 \) (strings where \(|x| \ne |y|\)):
   \[
   \begin{aligned}
   S_1 &\rightarrow L_3 \mid L_4 \\
   L_3 &\rightarrow C\ D\ \#\ D \\
   L_4 &\rightarrow D\ \#\ C\ D
   \end{aligned}
   \]
   Here, \( C \) generates non-empty strings, ensuring \( |x| > |y| \) in \( L_3 \) and \( |x| < |y| \) in \( L_4 \).

Productions for \( C \) and \( D \):
   \[
   \begin{aligned}
   C &\rightarrow 0 \mid 1 \mid 0C \mid 1C \\
   D &\rightarrow \varepsilon \mid 0D \mid 1D
   \end{aligned}
   \]

Productions for \( S_2 \) (strings where \( x = y^R \)):
   \[
   S_2 \rightarrow 0\ S_2\ 0 \mid 1\ S_2\ 1 \mid \#
   \]
   This generates reverse strings with '\#' in the middle.


\( S \): Generates all strings in \( L \).


\( S_1 \): Generates strings \( x\#y \) where \( |x| \ne |y| \).


\( S_2 \): Generates strings \( x\#y \) where \( x = y^R \).


\( L_3 \): Generates strings where \( x = CD \), \( y = D \), ensuring \( |x| > |y| \).


\( L_4 \): Generates strings where \( x = D \), \( y = CD \), ensuring \( |x| < |y| \).


\( C \): Generates all non-empty strings over \( \{0, 1\} \).


\( D \): Generates all strings (including empty string) over \( \{0, 1\} \).


The grammar is unambiguous. This is because: Strings where \( x = y^R \) are generated exclusively by \( S_2 \). Strings where \( |x| \ne |y| \) are generated exclusively by \( S_1 \). There is no overlap between \( S_1 \) and \( S_2 \) since \( x = y^R \) implies \( |x| = |y| \), and strings with \( |x| \ne |y| \) cannot be palindromes centered at '\#'. Therefore, every string in \( L \) has exactly one derivation in the grammar, we can draw the conclusion that the grammar is unambiguous.


\vspace{1cm}

(b) Let the CFG \( G = (V, \Sigma, R, S) \) be defined as follows:

Variables: \( V = \{ S \} \)


Terminals: \( \Sigma = \{ a, b, c \} \)


Start Variable: \( S \)


Production Rules \( R \):

  \[
  S \rightarrow a\ S\ c\ \mid\ b\ S\ c\ \mid\ \varepsilon
  \]


\( S \): Generates all strings in \( L \) where the number of \( c \)'s equals the total number of \( a \)'s and \( b \)'s. 

This structure guarantees that each 'a' or 'b' before the \( c \)'s is matched with a 'c' after the \( S \) in the production. Therefore, the total number of \( c \)'s (\( p \)) will always equal the number of \( a \)'s (\( m \)) plus the number of \( b \)'s (\( n \)).


The grammar is unambiguous. Each production choice corresponds directly to the next symbol in the string. There is no alternative way to derive a given string because the positions of 'a' and 'b' before the 'c's dictate the derivation path. The recursive nature of the grammar ensures a one-to-one mapping between strings and their derivations.

\vspace{1cm}

\noindent\textbf{Collaborators: None}

\end{document}