\documentclass[12 pt]{article}         
\usepackage{amsfonts, amssymb}
\usepackage{fancyhdr}
\usepackage{amsmath}

\oddsidemargin=-0.5cm                  
\setlength{\textwidth}{6.5in}          
\addtolength{\voffset}{-20pt}        		
\addtolength{\headsep}{25pt}

\setlength{\headheight}{27.2pt}
\addtolength{\topmargin}{-12.7pt}

\pagestyle{fancy}
\fancyhf{}
\fancyhead[L]{Theory and Practice of Algorithms \\ Homework 1}
\fancyhead[R]{Jingheng Huan \\ \today}
\fancyfoot[C]{\thepage}

\begin{document}

\section*{Proposition 1}	

Prove that $\forall P, Q \in \mathbb{B}.(P \implies Q) \iff (\neg P \lor Q)$.

\noindent\textbf{Proof:} \\[10pt]
We need to prove this equivalence by proving both directions. \\
Let's start with: \\
\( P \implies Q \Rightarrow \neg P \lor Q \) \\
Assume \( P \implies Q \) holds. By the definition of implication, this means that:
\begin{itemize}
    \item If \( P \) is true, then \( Q \) is true.
    \item If \( P \) is false, the implication holds regardless of \( Q \)'s truth value.
\end{itemize}
Now, we show that \( \neg P \lor Q \) holds by considering two cases:
\begin{itemize}
    \item \textbf{Case 1:} \( P \) is false. In this case, \( \neg P \) is true, and thus \( \neg P \lor Q \) holds regardless of the value of \( Q \).
    \item \textbf{Case 2:} \( P \) is true. Since \( P \implies Q \), we know \( Q \) must also be true. Therefore, \( \neg P \lor Q \) holds because \( Q \) is true.
\end{itemize}
Thus, we have shown that \( P \implies Q \Rightarrow \neg P \lor Q \). \\
Now let's continue to prove \( \neg P \lor Q \Rightarrow P \implies Q \) \\
Assume \( \neg P \lor Q \) holds. We will show that \( P \implies Q \) follows by considering two cases:
\begin{itemize}
    \item \textbf{Case 1:} \( P \) is false. In this case, \( \neg P \) is true, and therefore \( \neg P \lor Q \) holds regardless of the value of \( Q \). Since \( P \) is false, \( P \implies Q \) is true.
    \item \textbf{Case 2:} \( P \) is true. Since \( \neg P \lor Q \) holds, and \( P \) is true, \( Q \) must also be true. Therefore, \( P \implies Q \) holds.
\end{itemize}
Thus, we have shown that \( \neg P \lor Q \Rightarrow P \implies Q \).

Since we have proven both directions, we conclude that:
\[
P \implies Q \iff \neg P \lor Q
\]
\vspace{20pt}

\section*{Proposition 2}							
Prove that \(\forall n \in \mathbb{N}. \sum_{i=0}^{n} i^3 = \frac{1}{4} n^2 (n+1)^2\).

\noindent\textbf{Proof:} \\[10pt]              
We will use mathematical induction on $n$ to prove this statement.

\noindent\textbf{Base Case:}
For \(n = 0\), the left-hand side is \(\sum_{i=0}^{0} i^3 = 0^3 = 0\).  
The right-hand side is \(\frac{1}{4} \times 0^2 \times (0+1)^2 = 0\).  
Thus, the base case holds.

\vspace{10pt}

\noindent\textbf{Inductive Step:}
Assume that the statement is true for some \(k \in \mathbb{N}\),
\[
\sum_{i=0}^{k} i^3 = \frac{1}{4} k^2 (k+1)^2
\]
We need to prove that the statement is true for \(k+1\),
\[
\sum_{i=0}^{k+1} i^3 = \frac{1}{4} (k+1)^2 (k+2)^2
\]
Starting from the inductive hypothesis:
\[
\sum_{i=0}^{k+1} i^3 = \left(\sum_{i=0}^{k} i^3\right) + (k+1)^3
\]
Substitute the inductive hypothesis:
\[
\sum_{i=0}^{k+1} i^3 = \frac{1}{4} k^2 (k+1)^2 + (k+1)^3
\]
Factor out \((k+1)^2\) from the terms:
\[
\sum_{i=0}^{k+1} i^3 = (k+1)^2 \left(\frac{1}{4} k^2 + (k+1)\right)
\]
Simplify the expression inside the parentheses:
\[
\sum_{i=0}^{k+1} i^3 = (k+1)^2 \left(\frac{k^2 + 4k + 4}{4}\right) 
\]
\[
= (k+1)^2 \left(\frac{(k+2)^2}{4}\right)
\]
\[
= \frac{1}{4} (k+1)^2 (k+2)^2
\]
Thus, the inductive step holds. By the principle of mathematical induction, the statement is true for all \(n \in \mathbb{N}\).

\vspace{20pt}
\noindent\textbf{Collaborators: None}

\end{document}