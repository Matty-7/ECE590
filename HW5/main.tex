\documentclass[12 pt]{article}         
\usepackage{amsfonts, amssymb}
\usepackage{fancyhdr}
\usepackage{amsmath}
\usepackage{fancyhdr}
\usepackage{verbatimbox}
\usepackage{tikz}
\usetikzlibrary{automata, positioning}

\oddsidemargin=-0.5cm                  
\setlength{\textwidth}{6.5in}          
\addtolength{\voffset}{-20pt}        		
\addtolength{\headsep}{25pt}

\setlength{\headheight}{27.2pt}
\addtolength{\topmargin}{-12.7pt}

\pagestyle{fancy}
\fancyhf{}
\fancyhead[L]{Theory and Practice of Algorithms \\ Homework 5}
\fancyhead[R]{Jingheng Huan \\ \today}
\fancyfoot[C]{\thepage}

\begin{document}

\section*{Problem 1}

This is the sequence of configurations that the Turing Machine \(M_1\) enters when started on the input \(110\#11\).

\[
(q_1, \stackrel{\downarrow}{1}10\#11)
\]
\[
(q_3, x\stackrel{\downarrow}{1}0\#11)
\]
\[
(q_3, x1\stackrel{\downarrow}{0}\#11)
\]
\[
(q_3, x10\stackrel{\downarrow}{\#}11)
\]
\[
(q_5, x10\#\stackrel{\downarrow}{1}1)
\]
\[
(q_6, x10\stackrel{\downarrow}{\#}x1)
\]
\[
(q_7, x1\stackrel{\downarrow}{0}\#x1)
\]
\[
(q_7, x\stackrel{\downarrow}{1}0\#x1)
\]
\[
(q_7, \stackrel{\downarrow}{x}10\#x1)
\]
\[
(q_1, x\stackrel{\downarrow}{1}0\#x1)
\]
\[
(q_3, xx\stackrel{\downarrow}{0}\#x1)
\]
\[
(q_3, xx0\stackrel{\downarrow}{\#}x1)
\]
\[
(q_5, xx0\#\stackrel{\downarrow}{x}1)
\]
\[
(q_5, xx0\#x\stackrel{\downarrow}{1})
\]
\[
(q_6, xx0\#\stackrel{\downarrow}{x}x)
\]
\[
(q_6, xx0\stackrel{\downarrow}{\#}xx)
\]
\[
(q_7, xx\stackrel{\downarrow}{0}\#xx)
\]
\[
(q_7, x\stackrel{\downarrow}{x}0\#xx)
\]
\[
(q_1, xx\stackrel{\downarrow}{0}\#xx)
\]
\[
(q_2, xxx\stackrel{\downarrow}{\#}xx)
\]
\[
(q_4, xxx\#\stackrel{\downarrow}{x}x)
\]
\[
(q_4, xxx\#x\stackrel{\downarrow}{x})
\]

Here we can see that it stcuks in q4 state and can't go to the state of $q_accept$. Hence 110\#11 goes to reject state.


\vspace{1cm}

\section*{Problem 2}

(a) 
\textbf{Scan for ‘\#’}: Move right until the ‘\#’ symbol is found. If ‘\#’ is not found, reject.


\textbf{Compare symbols symmetrically}: Repeat until all symbols before '\#' are processed:
    \begin{enumerate}
        \item Move left from '\#' to find the first '0' or '1'), replace with 'X', and remember its value.
        \item Move right to '\#', then move right to find the first '0' or '1' after '\#'.
        \item Compare this symbol with the remembered value:
        \begin{itemize}
            \item If they match, mark it.
            \item If they do not match, then reject.
        \end{itemize}
    \end{enumerate}

\textbf{Check for extra symbols}: After all symbols before '\#' are marked, move right from '\#' to check for any unmarked symbols. If unmarked symbols are found, reject; else, accept.

\vspace{1cm}

(b)
\noindent The Turing machine \( M \) is defined as follows:


\( Q = \{ q_0, q_1, q_2, q_3, q_{\text{accept}}, q_{\text{reject}} \} \) is the set of states.


\( \Sigma = \{ 0, 1, \# \} \) is the input alphabet.


\( \Gamma = \{ 0, 1, \#, \_ \} \) is the tape alphabet, where \( \_ \) represents the blank symbol.


\( q_0 \) is the initial state.


\( q_{\text{accept}} \) is the accept state.


\( q_{\text{reject}} \) is the reject state.


\( \delta \) is the transition function, defined as follows:


\begin{center}
\begin{tabular}{|c|c|c|c|c|}
\hline
Current State & Current Symbol & New Symbol & Direction & New State \\
\hline
$q0$ & $0$   & $X$   & $R$ & $q1$ \\
$q0$ & $1$   & $X$   & $R$ & $q2$ \\
$q0$ & $\#$  & $\#$  & $R$ & $q6$ \\
\hline
$q1$ & $0$   & $0$   & $R$ & $q1$ \\
$q1$ & $1$   & $1$   & $R$ & $q1$ \\
$q1$ & $X$   & $X$   & $R$ & $q1$ \\
$q1$ & $\#$  & $\#$  & $R$ & $q3$ \\
\hline
$q2$ & $0$   & $0$   & $R$ & $q2$ \\
$q2$ & $1$   & $1$   & $R$ & $q2$ \\
$q2$ & $X$   & $X$   & $R$ & $q2$ \\
$q2$ & $\#$  & $\#$  & $R$ & $q5$ \\
\hline
$q3$ & $X$   & $X$   & $R$ & $q3$ \\
$q3$ & $0$   & $X$   & $L$ & $q4$ \\
\hline
$q4$ & $0$   & $0$   & $L$ & $q4$ \\
$q4$ & $1$   & $1$   & $L$ & $q4$ \\
$q4$ & $X$   & $X$   & $L$ & $q4$ \\
$q4$ & $\#$  & $\#$  & $L$ & $q4$ \\
$q4$ & $\_$  & $\_$  & $R$ & $q0$ \\
\hline
$q5$ & $X$   & $X$   & $R$ & $q5$ \\
$q5$ & $1$   & $X$   & $L$ & $q4$ \\
\hline
$q6$ & $X$   & $X$   & $R$ & $q6$ \\
$q6$ & $\_$  & $\_$  & $S$ & $q\_accept$ \\
\hline
\end{tabular}
\end{center}


\noindent\textbf{Collaborators: None}

\end{document}